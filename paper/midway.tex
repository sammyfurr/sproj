\documentclass[12pt]{article}
\usepackage{multicol}
\usepackage{cite}

\author{Sammy Furr}
\title{Teaching Debugging Collaboratively---Midway Report}
\date{\today}

\begin{document}

\begin{titlepage}
\maketitle
\end{titlepage}

\section{Introduction}

Debugging is invaluable in writing and understanding code, yet it is
rarely formally taught\cite{doi:10.1080/08993400802114581}.  We
typically teach students programming structures, concepts, and
languages, but we leave them to learn the tools they use to code by
themselves.  This approach often works well---the programmer's choice
of editor is \textit{very} personal, students figure out how to
configure an individualized workflow.  Perhaps because debuggers are
tools, they often get lumped into the ``teach yourself'' category.
Unlike editors or reference guides however, effectively using a
debugger requires a set of high-level, platform agnostic, teachable
skills.  Teaching these skills is effective, and translates into
better, faster, debugging and
programming\cite{10.1145/3286960.3286970}\cite{10.1145/3361721.3361724}.\par

The use of a debugger is particularly important in a low-level
programming classes, which is often the first time that students
encounter concepts like assembly instructions and memory addresses.
Debuggers such as GCC offer incredibly powerful tools to step through
and inspect running code.  Sadly, low level debugging tools are often
less than intuitive, and provide little to no means for collaboration.
Powerful tools such as RR exist that enable collaborative ``record and
replay'' debugging\cite{DBLP:journals/corr/OCallahanJFHNP17}, but they
lack tools to visualize address spaces and control flow of programs.
Research shows that visualization of these previously unexplored
spaces aids students taking low-level or systems programming
classes\cite{10.1145/3328778.3366894}.\par

The necessity for accessible, collaborative tools for teaching all
aspects of computer science, not the least debugging, has been
exemplified by the current COVID-19 crisis.  Many students don't have
access to a high powered computer (or even a computer at all) to run
tools that allow for collaboration and replay by capturing the entire
state of a virtual machine\cite{10.1145/2843859.2843867}.  Though the
limitations of interacting with a debugger through a phone or
low-powered laptop seem daunting, I believe they are actually
constraints that can help better shape a visualized, collaborative
approach.\par

For my senior project, I want to develop a front-end interface to RR
that enables collaborative debugging and process visualization on a
variety of non-traditional platforms.

\section{The Value of Teaching Debugging}

Before creating a platform for teaching debugging it's of course
important to make sure that teaching debugging is actually valuable.
There is not a lot research specifically into the efficacy of teaching
debugging for computer science students, despite a recent rise in the
inclusion of debugging in ``computational thinking''
curriculums\cite{10.1145/3361721.3361724}.  These curriculums attempt
to teach skills in computer science classes that are useful in other
subject areas: the UK's computer science curriculum considers
debugging an essential ``transferable skill''\cite{10.1145/2602484}.
Though there seems to be confidence that the problem-solving
techniques used in debugging are widely applicable, I am more
interested in whether systematically teaching debugging actually
benefits computer science students.  Michaeli and Romeike conducted a
good, albeit somewhat small, study on the efficacy of teaching a
systematic debugging process to K12 students.  They found that
students who have been taught a specific debugging framework performed
better in debugging tests and were more confident in their own
debugging skills\cite{10.1145/3361721.3361724}.  Their result is
positive evidence towards the efficacy of teaching debugging, though
their research doesn't include college or university students.\par

As Michaeli and Romeike point out, there is a lack of research into
the value of teaching debugging in higher education.

Research into how to best teach debugging is self-admittedly sparse.
Chan et al. allow that ``in general research on how to improve
debugging is sporadic''---an observation that leads them to research a
framework to reduce the complexity of teaching
debugging\cite{10.1145/3286960.3286970}.  To organize their framework,
they split debugging knowledge into 5 categories: Domain, System,
Procedural, Strategic, and Experiential.  They then review different
debugging tools and teaching aids---from those that involve writing
code to games---and map tools to the knowledge areas they seek to
address.  After an evaluation of a host of different tools, they claim
a few significant faults in current debugging teaching platforms.  The
two of which I seek to address follow:
\begin{itemize}
\item A lack of tools addressing system knowledge (an understanding of
  the program to be debugged).
\item A lack of back-tracing ability/coverage.
\end{itemize}

\section{Debugger: Low Level}

\section{Debugger: High Level}

\section{Next Steps}

\pagebreak
\bibliographystyle{acm}
\bibliography{sprojbib}{}
\end{document}
