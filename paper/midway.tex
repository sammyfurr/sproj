\documentclass[12pt, twocolumn]{article}
\usepackage{multicol}
\usepackage{cite}

\author{Sammy Furr}
\title{Teaching Debugging Collaboratively---Midway Report}
\date{\today}

\begin{document}

\begin{titlepage}
\maketitle
\end{titlepage}

\section{Introduction}

Debugging is invaluable in writing and understanding code, yet it is
rarely formally taught.  We typically teach students programming
structures, concepts, and languages, but we leave them to learn the
tools they use to code by themselves.  This approach often works
well---the programmer's choice of editor is \textit{very} personal,
students figure out how to configure an individualized workflow.
Perhaps because debuggers are tools, they often get lumped into the
``teach yourself'' category.  Unlike editors or reference guides
however, effectively using a debugger requires a set of high-level,
platform agnostic, teachable skills.  Teaching these skills is
effective, and translates into better, faster, debugging and
programming.\cite{10.1145/3286960.3286970}\cite{10.1145/3361721.3361724}\par

The use of a debugger is particuarly important in a low-level
programming classe, which is often the first time that students
encounter concepts like assembly instructions and memory addresses.
Debuggers such as GCC offer incredibly powerful tools to step through
and inspect running code.  Sadly, low level debugging tools are often
less than intuitive, and provide little to no means for collaboration.
Powerful tools such as RR exist that enable collaborative ``record and
replay'' debugging\cite{DBLP:journals/corr/OCallahanJFHNP17}, but they
lack tools to visualize address spaces and control flow of programs.
Research shows that visualization of these previously unexplored
spaces aids students taking low-level or systems programming
classes\cite{10.1145/3328778.3366894}.\par

The necessity for accessible, collaborative tools for teaching all
aspects of computer science, not the least debugging, has been
exemplified by the current COVID-19 crisis.

\section{The Value of Teaching Debugging}

\section{Debugger: Low Level}

\section{Debugger: High Level}

\section{Next Steps}

\pagebreak
\bibliographystyle{acm}
\bibliography{sprojbib}{}
\end{document}
